%%%
% Plantilla de Memoria
% Modificación de una plantilla de Latex de Nicolas Diaz para adaptarla 
% al castellano y a las necesidades de escribir informática y matemáticas.
%
% Editada por: Mario Román
%
% License:
% CC BY-NC-SA 3.0 (http://creativecommons.org/licenses/by-nc-sa/3.0/)
%%%

%%%%%%%%%%%%%%%%%%%%%%%%%%%%%%%%%%%%%%%%%
% Thin Sectioned Essay
% LaTeX Template
% Version 1.0 (3/8/13)
%
% This template has been downloaded from:
% http://www.LaTeXTemplates.com
%
% Original Author:
% Nicolas Diaz (nsdiaz@uc.cl) with extensive modifications by:
% Vel (vel@latextemplates.com)
%
% License:
% CC BY-NC-SA 3.0 (http://creativecommons.org/licenses/by-nc-sa/3.0/)
%
%%%%%%%%%%%%%%%%%%%%%%%%%%%%%%%%%%%%%%%%%

%----------------------------------------------------------------------------------------
%   PAQUETES Y CONFIGURACIÓN DEL DOCUMENTO
%----------------------------------------------------------------------------------------

%%% Configuración del papel.
% microtype: Tipografía.
% mathpazo: Usa la fuente Palatino.
\documentclass[a4paper, 11pt]{article}
\usepackage[protrusion=true,expansion=true]{microtype}
\usepackage{mathpazo}
\usepackage{amsthm}
\usepackage{graphicx}




% Indentación de párrafos para Palatino
\setlength{\parindent}{0pt}
  \parskip=8pt
\linespread{1.05} % Change line spacing here, Palatino benefits from a slight increase by default


%%% Castellano.
% noquoting: Permite uso de comillas no españolas.
% lcroman: Permite la enumeración con numerales romanos en minúscula.
% fontenc: Usa la fuente completa para que pueda copiarse correctamente del pdf.
\usepackage[spanish,es-noquoting,es-lcroman]{babel}
\usepackage[utf8]{inputenc}
\usepackage[T1]{fontenc}
\selectlanguage{spanish}


%%% Gráficos
\usepackage{graphicx} % Required for including pictures
\usepackage{wrapfig} % Allows in-line images
\usepackage[usenames,dvipsnames]{color} % Coloring code


%%% Matemáticas
\usepackage{amsmath}


%%% Bibliografía
\makeatletter
\renewcommand\@biblabel[1]{\textbf{#1.}} % Change the square brackets for each bibliography item from '[1]' to '1.'
\renewcommand{\@listI}{\itemsep=0pt} % Reduce the space between items in the itemize and enumerate environments and the bibliography



%----------------------------------------------------------------------------------------
%   TÍTULO
%----------------------------------------------------------------------------------------
% Configuraciones para el título.
% El título no debe editarse aquí.
\renewcommand{\maketitle}{
  \begin{flushright} % Right align
  
  {\LARGE\@title} % Increase the font size of the title
  
  \vspace{50pt} % Some vertical space between the title and author name
  
  {\large\@author} % Author name
  \\\@date % Date
  \vspace{40pt} % Some vertical space between the author block and abstract
  \end{flushright}
}

%% Título
\title{\textbf{Backtracking}\\ % Title
} % Subtitle

\author{\textsc{Ignacio Mas Mesa,\\Braulio Valdivielso Martínez} % Author
\\{\textit{Universidad de Granada}}} % Institution

\date{\today} % Date



%----------------------------------------------------------------------------------------
%   DOCUMENTO
%----------------------------------------------------------------------------------------

\begin{document}

\maketitle % Print the title section

%% Resumen (Descomentar para usarlo)
\renewcommand{\abstractname}{Abstract} % Uncomment to change the name of the abstract to something else
\begin{abstract}
En esta práctica exploraremos la técnica conocida en algorítmica como backtracking. Desarrollaremos una solución para el problema propuesto usando esta técnica y mediremos su rendimiento de forma empírica frente a otra solución que además de backtracking utiliza poda \(\textit{branch and bound}\).
\end{abstract}

\vspace{30pt} % Some vertical space between the abstract and first section


%% Índice
  \tableofcontents

\pagebreak

%%% Inicio del documento

\section{Problema}
El problema que se nos ha propuesto es el siguiente:
\begin{quote}
Se desea dividir un conjunto de $n$ personas para formar dos equipos que competirán entre sí.
Cada persona tiene un cierto nivel de competición, que viene representado por una puntuación
(un valor numérico entero). Con el objeto de que los dos equipos tengan una capacidad de
competición similar, se pretende construir los equipos de forma que la suma de las puntuaciones
de sus miembros sea la misma. 
\end{quote}

\section{Soluciones}

Es bueno darse cuenta antes de empezar a trabajar que en realidad este problema
es completamente equivalente al \textit{subset problem}, i.e., encontrar si existe
un subconjunto de un conjunto dado que sume $k$. Esto nos facilita la labor ya
que nos da igual uno de los dos equipos, sólo tenemos que trabajar con uno (el subconjunto
buscado) e ir aproximándonos al valor $k$, que en nuestro caso será la mitad de la 
suma total de la puntuación de todos jugadores.

\subsection{Backtracking}
Para realizar el trabajo que se nos pedía en la práctica realizamos en primer lugar un algoritmo ingenuo utilizando backtracking. Dado que el algoritmo ha de probar todos las posibles divisiones en equipos, se ejecuta con un orden de eficiencia temporal de $O(2^n)$. 

Debido a su poca eficiencia, este algoritmo resulta inviable para resolver el problema de forma seria, por lo que se hace necesario incorporar mejoras. De eso nos encargaremos en la próxima sección.

\subsection{Poda (\protect\textit{Branch \& Bound})}
El algoritmo de poda nace de la siguiente observación: si conseguimos averiguar de alguna manera que la solución que estamos explorando no puede llegar a ser la óptima, ¿para qué seguir explorándola? Darse cuenta de eso puede significar que los tiempos de ejecución disminuyan considerablemente.

En este caso, las soluciones parciales serán vectores $(x_1, ..., x_i)$ con
$x_i \in \{0, 1\}$, donde 0 representa que el jugador i-ésimo no se encuentra en la solución parcial
y 1 que sí. En cada paso comprobaremos qué ocurre cuando a dicha solución parcial le añadimos un 0
o un 1, es decir, si metemos el (i+1)-ésimo jugador en el equipo o no. 

Ahora bien, si ese nuevo equipo tiene más de la mitad de puntuación podemos descartarlo directamente
y seguir por otra rama. Si esto ocurriera, significaría que el equipo \"contrario\" tendría menos de la 
mitad de la puntuación total, por lo que esa solución sí la consideraríamos. Además, seguir explorando esa 
rama sólo podría aumentar la diferencia de puntuación entre ambos equipos, justo lo contrario de lo
que buscamos.

\section{Comparación de rendimiento}

Para comparar los algoritmos vamos a ejecutarlos con sets de datos de 5 a 50 elementos
y medir el tiempo de ejecución. Antes de esto, cabe esperar que la segunda versión sea
mejor, ya que evita muchas de las comprobaciones. Efectivamente, así se comprueba en los 
datos obtenidos:

\begin{center}
\begin{tabular}{|c|c|c|}

\hline
Tamaño del problema & Backtracking & Branch \& Bound \\
\hline
\hline
5 & 0.0000646430 & 0.0000031570 \\
\hline
10 & 0.0003192760 & 0.0000113580 \\
\hline
15 & 0.0008492850 & 0.0000099330 \\
\hline
20 & 0.0051420740 & 0.0000070650 \\
\hline
25 & 0.0137037790 & 0.0000064230 \\
\hline
30 & 0.1462037010 & 0.0000229280 \\
\hline
35 & 0.7887776860 & 0.0000099950 \\
\hline
40 & 10.5693192850 & 0.0000182760 \\
\hline
45 & 25.9811328990 & 0.0000089000 \\
\hline
50 & 55.9817939720 & 0.0000096890 \\
\hline
\end{tabular}
\end{center}

\includegraphics[scale=0.7]{data/graph.jpg}

Como se puede observar, el primer algoritmo ni siquiera se acerca al otro en cuanto
a eficiencia y velocidad. Esto se debe, como ya hemos comentado, a la gran cantidad
de casos inútiles que el primero procesa mientras el segundo se los ahorra.

Esto se ve más fácilmente si pensamos en el último ejemplo de los 50 jugadores.
Suponiendo una distribución más o menos uniforme de las puntuaciones, metidos 25 jugadores
en un equipo las puntuaciones se acercarían mucho. Si, por ejemplo, necesitáramos hasta
30 jugadores para pasarnos de la mitad, aun así nos ahorraríamos $2^{20} = 1048576$ casos
que el otro algoritmo habría de comprobar. Esto, además, se produciría para cada
equipo de 30 jugadores, esto es, ${50 \choose 30} = 47129212243960$. Obviamente esto es 
sólo una aproximación al caso real y los números no son todo lo precisos que deberían,
pero sirven para hacernos una idea de cuánto mejor es el segundo algoritmo respecto del
primero.

\section{Conclusiones}

Como hemos visto a lo largo de esta memoria, los algoritmos que usan poda resultan
mucho mejores que sus homólogos básicos, ya que consigues ahorrarse el procesamiento
de gran cantidad de casos que, de todas formas, no iban a resultar de utilidad a la hora
de encontrar una solución. Por eso, siempre debemos intentar ver qué condiciones de poda
podemos aplicar a un cierto problema mientras buscamos las soluciones.

\end{document}